\section{Introduction}

    \subsection{What are multilevel models?}
    \begin{frame}
        \frametitle{\insertsection\ $-$ \insertsubsection}
        \begin{itemize}
            \item Hierarchical data are common in observational data, with individuals nested within geographical areas or institutions such as schools or workplaces.
            \item Multilevel models recognise the existence of such data hierarchies by allowing for residual components at each level in the hierarchy.
            \item This model would partition the residual variance at different levels of the hierarchy, such as between-group and within-group components, capturing unobserved characteristics that affect outcomes.
            \item Reference: \url{https://www.bristol.ac.uk/cmm/learning/multilevel-models/what-why.html}
        \end{itemize}
    \end{frame}

    \subsection{Statistical model}
    \begin{frame}
        \frametitle{\insertsection\ $-$ \insertsubsection}
        Linear multilevel model with 1 predictor for individual $i$ in group $j$:
        \begin{equation}
            Y_{ij} = \underbrace{(\beta_0 + u_{0j})}_{\beta_{0j}} + \underbrace{(\beta_1 + u_{1j})}_{\beta_{1j}} X_{1,ij} + \epsilon_{ij}
        \end{equation}
        \begin{itemize}
            \item $Y_{ij}$: Dependent variable for individual $i$ in group $j$
            \item $X_{1,ij}$: Predictor variable 1 for individual $i$ in group $j$
            \item $\beta_{0j}$: Intercept for group $j$, which consists of:
                \begin{itemize}
                    \item $\beta_0$: Overall intercept across all groups
                    \item $u_{0j} \sim \mathcal{N}(0, \sigma_{u_0}^2)$: Random effect of intercept for group $j$
                \end{itemize}
            \item $\beta_{1j}$: Slope for predictor $X_{1,ij}$ for group $j$, which consists of:
                \begin{itemize}
                    \item $\beta_1$: Overall slope for predictor $X_{1,ij}$ across all groups
                    \item $u_{1j} \sim \mathcal{N}(0, \sigma_{u_1}^2)$: Random effect of slope for predictor $X_{1,ij}$ for group $j$
                \end{itemize}
            \item $\epsilon_{ij} \sim \mathcal{N}(0, \sigma_\epsilon^2)$: Residual error for individual $i$ in group $j$
        \end{itemize}
        \textcolor{red}{Notes for presentation: its possible to add predictor $X_{2,ij}$ or $Z_1j$ and so on (comment later)}
    \end{frame}

    \subsection{Why use multilevel models?}
    \begin{frame}
        \frametitle{\insertsection\ $-$ \insertsubsection}
        \begin{enumerate}
            \item[1.] \textbf{Correct Inferences}: Traditional methods assume independent observations, which are often false. Additionally, ignoring hierarchical structures can lead to underestimated standard errors and overstated statistical significance.
            \item[2.] \textbf{Group Effects Estimation}: Directly quantify between-group variation and identify outlying groups.
            \item[3.] \textbf{Simultaneous Estimation}: Unlike fixed effects models, the separation of observed and unobserved group characteristics is possible, allowing for simultaneous estimation of group-level and individual-level effects.
            \item[4.] \textbf{Generalization Beyond Sample}: Unlike fixed effects models which only describe sampled groups, multilevel models treat groups as random samples from a population, enabling generalizations to unobserved groups.
            \item[5.] Reference: \url{https://www.bristol.ac.uk/cmm/learning/multilevel-models/what-why.html}
        \end{enumerate}
    \end{frame}

    \subsection{Issues with large multilevel datasets}
    \begin{frame}
        \frametitle{\insertsection\ $-$ \insertsubsection}
        \begin{itemize}
            \item Hierarchical data expands dramatically, potentially reaching gigabytes in size ($N$ groups $\times$ $n$ individuals $=$ massive datasets).
            \item \textbf{Large number of groups}: Computational bottleneck from numerical integration over random effects for each group at every optimization step, leading to high computational costs.
            \item \textbf{Large group sizes}: High-dimensional multivariate distributions create numerical issues with large covariance matrix inversion, even in linear models.
            % \item \textbf{Large groups and sizes} (large $N$, large $n$): Combines both of the above issues, resulting in extremely high computational demands and numerical challenges.
            \item Reference: \href{https://m-clark.github.io/posts/2019-10-20-big-mixed-models/}{Clark, Michael. (2019).}, Speelman, Dirk. (2018) chapter 2.
        \end{itemize}
    \end{frame}