\section{Introduction}

    \subsection{What \& Why}
    \begin{frame}
        \frametitle{\insertsection\ $-$ \insertsubsection}
        \begin{columns}[c] 
            \begin{column}{0.48\textwidth}
                \textbf{What are multilevel models?}
                \begin{itemize}
                    \item Hierarchical Structure
                    \item Residual Components
                    \item Variance Partitioning
                \end{itemize}
            \end{column}
            \begin{column}{0.5\textwidth}
                \textbf{Why use multilevel models?}
                \begin{itemize}
                    \item Correct Inferences
                    \item Group Effects Estimation
                    \item Simultaneous Estimation
                    \item Generalization Beyond Sample \footnotemark
                \end{itemize}
            \end{column}
        \end{columns} 
        \footcitetext{bristolCMM}

        \vspace{0.4cm}

        \textbf{Statistical Model}
        \[
            \begin{aligned}
                y_n &\sim \mathcal{N}\left(\mu_n, \sigma\right) \\
                \mu_n &= b_0 + \sum_{p=1}^P b_p x_{pn} + \tilde{b}_{0j[n]} + \sum_{p=1}^P \tilde{b}_{pj} x_{pn}
            \end{aligned}
        \]
    \end{frame}

    \note{
        \tiny
        \textbf{What are multilevel models?}
        \begin{itemize}
            \item Hierarchical Structure: Observational data often feature individuals nested within higher-level groups, such as schools, workplaces, or geographical areas.
            \item Residual Components: Multilevel models account for these hierarchies by incorporating residual components at every level of the data structure.
            \item Variance Partitioning: These models divide residual variance into between-group and within-group components to capture unobserved factors influencing outcomes.
        \end{itemize}
        \textbf{Why use multilevel models?}
        \begin{itemize}
            \item Correct Inferences: Traditional methods assume independent observations, which are often false. Additionally, ignoring hierarchical structures can lead to underestimated standard errors and overstated statistical significance.
            \item Group Effects Estimation: Directly quantify between-group variation and identify outlying groups.
            \item Simultaneous Estimation: Unlike fixed effects models, the separation of observed and unobserved group characteristics is possible, allowing for simultaneous estimation of group-level and individual-level effects.
            \item Generalization Beyond Sample: Unlike fixed effects models which only describe sampled groups, multilevel models treat groups as random samples from a population, enabling generalizations to unobserved groups.
        \end{itemize}
    }
    \note{
        \tiny
        \textbf{Statistical Model}
        \[
            \begin{aligned}
                y_n &\sim \mathcal{N}\left(\mu_n, \sigma\right) \\
                \mu_n &= b_0 + b_1 x_{1n} + \ldots + b_p x_{pn} + \tilde{b}_{0j[n]} + \tilde{b}_{1j[n]} x_{1n} + \ldots + \tilde{b}_{pj[n]} x_{pn}
            \end{aligned}
        \]
        where:
        \begin{itemize}
            \item $y_n$: dependent variable for observation $n$
            \item $x_{pn}$: predictor variable $p$ for observation $n$
            \item $b_p$: overall slope (or intercept for $p=0$) for predictor $p$ across all groups
            \item $\tilde{b}_{pj[n]}$: random effect of predictor $p$ for group $j$ that observation $n$ belongs to
            \item $\sigma$: residual standard deviation (assumed constant across observations)
        \end{itemize}
    }

    \subsection{Issues with large datasets}
    \begin{frame}
        \frametitle{\insertsection\ $-$ \insertsubsection}
        \begin{itemize}
            \item Hierarchical data expands dramatically, potentially reaching gigabytes in size ($N$ groups $\times$ $n$ individuals $=$ massive datasets).
            \item \textbf{Large number of groups}: Computational bottleneck from numerical integration over random effects for each group at every optimization step, leading to high computational costs.
            \item \textbf{Large group sizes}: High-dimensional multivariate distributions create numerical issues with large covariance matrix inversion, even in linear models.\footcites{clark2019}{speelman2018}
        \end{itemize}
    \end{frame}