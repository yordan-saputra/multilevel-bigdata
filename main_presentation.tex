\documentclass{beamer}
\setbeamertemplate{caption}[numbered]
\usepackage[utf8]{inputenc}
\usepackage{xcolor}

\usetheme{Madrid}
\usecolortheme{default}

% CONFIGURATION FOR THE TITLE PAGE---------------------------
\title[Multilevel Models for Big Data]{Multilevel Models for Big Data}
\subtitle{Approaches for handling very large datasets}

\author[Ömercan \& Yordan]{Ömercan Mısırlıoğlu, Yordan Saputra}
\institute[]
{
    Department of Statistics\\
    TU Dortmund University
}

\date[Winter Term 2025/2026]{Final Presentation TBA, January (?) 2026}
% END OF CONFIGURATION---------------------------------------

% CONFIGURATION FOR THE BEGINNING OF EACH SECTION------------
\AtBeginSection[]
{
    \begin{frame}
        \frametitle{Table of Contents}
        \begin{columns}[t]
            \begin{column}{0.48\textwidth}
                \tableofcontents[sections={1-2},currentsection]
            \end{column}
            \begin{column}{0.48\textwidth}
                \tableofcontents[sections={3-},currentsection]
            \end{column}
        \end{columns}
        % \tableofcontents[currentsection]
    \end{frame}
}
% END OF CONFIGURATION---------------------------------------

% OTHER CONFIGURATION --------------------------------------
\setbeamertemplate{itemize item}{\textbullet}
% END OF CONFIGURATION---------------------------------------

\begin{document}

    \frame{\titlepage}

    \begin{frame}
        \frametitle{IMPORTANT NOTES}
        \begin{itemize}
            \item DONT FORGET TO CITE PROPERLY. rn im just adding links as references but we should have a proper bibliography slide at the end.
            \item \textcolor{red}{Notes for Omer: the blog \href{https://m-clark.github.io/posts/2019-10-20-big-mixed-models/}{Clark, Michael. (2019).} goes to talk about speed comparison between lme4 and mgcv (mainly for GAM, but can also be used for linear model). But, I dont think we should explore the GAM part since it is another topic? I want to stay on linear model. what do u think?}
            \item Add more notes here
        \end{itemize}
    \end{frame}

    \begin{frame}
        \frametitle{QUESTIONS FOR FEEDBACK}
        \begin{itemize}
            \item do we need to fit our own model using either real or syntethic data? or just explanation is enough?
            \item should we go in deeper on the dependent / overlapping subsamples?
            \item is the R packages part needed?
            \item also, im not sure whether the R packages here is using split sample approach or not. from what i've read, it doesnt seem like it, but this is the only source we have
            \item Add more questions here
        \end{itemize}
    \end{frame}

    \section{Introduction}

    \subsection{What are multilevel models?}
    \begin{frame}
        \frametitle{\insertsection\ $-$ \insertsubsection}
        \begin{itemize}
            \item Hierarchical data are common in observational data, with individuals nested within geographical areas or institutions such as schools or workplaces.
            \item Multilevel models recognise the existence of such data hierarchies by allowing for residual components at each level in the hierarchy.
            \item This model would partition the residual variance at different levels of the hierarchy, such as between-group and within-group components, capturing unobserved characteristics that affect outcomes.
            \item Reference: \url{https://www.bristol.ac.uk/cmm/learning/multilevel-models/what-why.html}
        \end{itemize}
    \end{frame}

    \subsection{Statistical model}
    \begin{frame}
        \frametitle{\insertsection\ $-$ \insertsubsection}
        Linear multilevel model with 1 predictor for individual $i$ in group $j$:
        \begin{equation}
            Y_{ij} = \underbrace{(\beta_0 + u_{0j})}_{\beta_{0j}} + \underbrace{(\beta_1 + u_{1j})}_{\beta_{1j}} X_{1,ij} + \epsilon_{ij}
        \end{equation}
        \begin{itemize}
            \item $Y_{ij}$: Dependent variable for individual $i$ in group $j$
            \item $X_{1,ij}$: Predictor variable 1 for individual $i$ in group $j$
            \item $\beta_{0j}$: Intercept for group $j$, which consists of:
                \begin{itemize}
                    \item $\beta_0$: Overall intercept across all groups
                    \item $u_{0j} \sim \mathcal{N}(0, \sigma_{u_0}^2)$: Random effect of intercept for group $j$
                \end{itemize}
            \item $\beta_{1j}$: Slope for predictor $X_{1,ij}$ for group $j$, which consists of:
                \begin{itemize}
                    \item $\beta_1$: Overall slope for predictor $X_{1,ij}$ across all groups
                    \item $u_{1j} \sim \mathcal{N}(0, \sigma_{u_1}^2)$: Random effect of slope for predictor $X_{1,ij}$ for group $j$
                \end{itemize}
            \item $\epsilon_{ij} \sim \mathcal{N}(0, \sigma_\epsilon^2)$: Residual error for individual $i$ in group $j$
        \end{itemize}
        \textcolor{red}{Notes for presentation: its possible to add predictor $X_{2,ij}$ or $Z_1j$ and so on (comment later)}
    \end{frame}

    \subsection{Why use multilevel models?}
    \begin{frame}
        \frametitle{\insertsection\ $-$ \insertsubsection}
        \begin{enumerate}
            \item[1.] \textbf{Correct Inferences}: Traditional methods assume independent observations, which are often false. Additionally, ignoring hierarchical structures can lead to underestimated standard errors and overstated statistical significance.
            \item[2.] \textbf{Group Effects Estimation}: Directly quantify between-group variation and identify outlying groups.
            \item[3.] \textbf{Simultaneous Estimation}: Unlike fixed effects models, the separation of observed and unobserved group characteristics is possible, allowing for simultaneous estimation of group-level and individual-level effects.
            \item[4.] \textbf{Generalization Beyond Sample}: Unlike fixed effects models which only describe sampled groups, multilevel models treat groups as random samples from a population, enabling generalizations to unobserved groups.
            \item[5.] Reference: \url{https://www.bristol.ac.uk/cmm/learning/multilevel-models/what-why.html}
        \end{enumerate}
    \end{frame}

    \subsection{Issues with large multilevel datasets}
    \begin{frame}
        \frametitle{\insertsection\ $-$ \insertsubsection}
        \begin{itemize}
            \item Hierarchical data expands dramatically, potentially reaching gigabytes in size ($N$ groups $\times$ $n$ individuals $=$ massive datasets).
            \item \textbf{Large number of groups}: Computational bottleneck from numerical integration over random effects for each group at every optimization step, leading to high computational costs.
            \item \textbf{Large group sizes}: High-dimensional multivariate distributions create numerical issues with large covariance matrix inversion, even in linear models.
            % \item \textbf{Large groups and sizes} (large $N$, large $n$): Combines both of the above issues, resulting in extremely high computational demands and numerical challenges.
            \item Reference: \href{https://m-clark.github.io/posts/2019-10-20-big-mixed-models/}{Clark, Michael. (2019).}, Speelman, Dirk. (2018) chapter 2.
        \end{itemize}
    \end{frame}
    \section{The split-sample approach}

    \subsection{Pseudo Likelihood}
    \begin{frame}
        \frametitle{\insertsection\ $-$ \insertsubsection}
        \begin{itemize}
            \item Consider the log-likelihood function $\ell(\boldsymbol{\theta}) = \sum_i \ell(\boldsymbol{y_i}|\boldsymbol{\theta})$ where $\boldsymbol{y_i}$ is the vector of all observations in group $i$
            \item Replaces the log-likelihood contribution $\ell(\boldsymbol{y_i}|\boldsymbol{\theta})$ by a weighted sum of log-likelihood contributions for sub-vectors $\boldsymbol{Y_i}^{(s)}$
            \item More specifically, the pseudo-log-likelihood function:
            \[
                p\ell(\boldsymbol{\psi}) = \sum_i \sum_s \delta_s\ \ell(\boldsymbol{y_i}^{(s)}|\boldsymbol{\psi})
            \]
            is maximized instead with respect to $\boldsymbol{\psi}$, which is not necessarily identical to $\boldsymbol{\theta}$
            \item Although $\hat{\boldsymbol{\psi}}$ is not the MLE estimate, it still has similar properties such as consistency and asymptotic normality \footcite{clark2019}
        \end{itemize}
    \end{frame}

    \note{
        \begin{itemize}
            \item Now, how do we split $\boldsymbol{y_i}$ into sub-vectors $\boldsymbol{Y_i}^{(s)}$? There are different ways to do this, and we will discuss some of them in the next slides.
            \item Going back to the slide on ``issues with large datasets'', it's obvious that we can split the data in two (technically three) different ways: either we can split the data by groups, or we can split the data by observations within groups. The first one is more suitable when we have a large number of groups, while the second one is more suitable when we have a large number of observations within groups.
        \end{itemize}
    }

    \subsection{Graphical representation}
    \begin{frame}
        \frametitle{\insertsection\ $-$ \insertsubsection}
        \begin{figure}
            \centering
            \includegraphics[width=0.73\textwidth]{split_image.png}
            \caption{Graphical representation of different ways to split large samples}
            \label{fig:1}
        \end{figure}
    \end{frame}

    \subsection{Independent subsamples}
    \begin{frame}
        \frametitle{\insertsection\ $-$ \insertsubsection}
        \begin{itemize}
            \item Shown in panel (b) of Figure~\ref{fig:1}, dataset with large $N$ is partitioned into $M$ independent sets $S_m$ of groups, where $m = 1, \ldots, M$
            \item In each subsample, the model is fitted, yielding an estimate $\hat{\boldsymbol{\theta}}_m$ of $\boldsymbol{\theta}$, equivalent to maximizing
            \[
                p\ell(\boldsymbol{\psi}) = \sum_m \sum_{i \in S_m} \ell(\boldsymbol{Y_i}|\boldsymbol{\theta}_m)
            \]
            with respect to $\boldsymbol{\psi} = \{\boldsymbol{\theta}_1, \ldots, \boldsymbol{\theta}_M\}$
            \item All $\boldsymbol{\theta}_m$ are equal to $\boldsymbol{\theta}$, therefore the estimates $\hat{\boldsymbol{\theta}}_m$ can be averaged to obtain an overall estimate $\hat{\boldsymbol{\theta}}$
        \end{itemize}
    \end{frame}

    \note{
        \begin{itemize}
            \item $\boldsymbol{\theta}_m$ are all equal to $\boldsymbol{\theta}$ because the subsamples are independent
            \item Mention parallelization here, since we can fit the model on each subsample in parallel, which can significantly reduce the computational time.
        \end{itemize}
    }

    \subsection{Dependent subsamples}
    \begin{frame}
        \frametitle{\insertsection\ $-$ \insertsubsection}
        \begin{itemize}
            \item Shown in panel (c) of Figure~\ref{fig:1}, dataset with large $n$ is partitioned into $M$ (not independent) sets $S_m$ of groups, where $m = 1, \ldots, M$
            \item Fitting the model on each subsample, equivalent to maximizing
            \[
                p\ell(\boldsymbol{\psi}) = \sum_m \sum_{i} \ell(\boldsymbol{Y_i}^{(m)}|\boldsymbol{\theta}_m)
            \]
            with respect to $\boldsymbol{\psi} = \{\boldsymbol{\theta}_1, \ldots, \boldsymbol{\theta}_M\}$, where $\boldsymbol{Y_i}^{(m)}$ is the observations in $\boldsymbol{Y_i}$ belonging to subsample $S_m$.
            \item All $\boldsymbol{\theta}_m$ are not necessarily equal to $\boldsymbol{\theta}$, therefore the combination of all $\hat{\boldsymbol{\theta}}_m$ into a single estimator $\hat{\boldsymbol{\theta}}$ depends on the precise model and data structure.
        \end{itemize}
    \end{frame}

    \note{
        \begin{itemize}
            \item $\boldsymbol{\theta}_m$ are not necessarily equal to $\boldsymbol{\theta}$ because the subsamples are not independent, and there may be some correlation between the observations in different subsamples.
            \item (GPT warning, dont trust 100\%) The combination of all $\hat{\boldsymbol{\theta}}_m$ into a single estimator $\hat{\boldsymbol{\theta}}$ can be done using various methods, such as meta-analysis techniques, or by fitting a model to the estimates $\hat{\boldsymbol{\theta}}_m$ themselves.
        \end{itemize}
    }

    \subsection{Overlapping subsamples}
    \begin{frame}
        \frametitle{\insertsection\ $-$ \insertsubsection}
        \begin{itemize}
            \item Shown in panel (d) of Figure~\ref{fig:1}, longitudinal dataset with very large $n$ is partitioned similarly to dependent subsamples, but association between longitudinal observations is accounted for by letting the subsamples overlap
            \item Denoting the parameters in pair $\{\boldsymbol{Y_i}^{(p)}, \boldsymbol{Y_i}^{(q)}\}$ by $\boldsymbol{\theta}_{p,q}$, fitting the models on all pairs is equivalent to maximizing
            \[
                p\ell(\boldsymbol{\psi}) = \sum_{p < q} \sum_{i} \ell(\boldsymbol{Y_i}^{(p)}, \boldsymbol{Y_i}^{(q)}|\boldsymbol{\theta}_{p,q})
            \]
            with respect to $\boldsymbol{\psi} = \{\boldsymbol{\theta}_{1,2}, \boldsymbol{\theta}_{1,3}, \ldots, \boldsymbol{\theta}_{Q-1,Q} \}$, where $\boldsymbol{Y_i}^{(p)}$ and $\boldsymbol{Y_i}^{(q)}$ are the observations in $\boldsymbol{Y_i}$ belonging to subsamples $S_p$ and $S_q$, respectively.
            \item Similarly, the combination of all $\hat{\boldsymbol{\theta}}_{p,q}$ into a single estimator $\hat{\boldsymbol{\theta}}$ depends on the precise model and data structure.
        \end{itemize}
    \end{frame}

    \note{
        \begin{itemize}
            \item without the pairwise fitting, we have to fit the model on the entire dataset, which is computationally infeasible when $n$ is very large. By fitting the model on pairs of subsamples, we can reduce the computational burden while still accounting for the association between longitudinal observations.
            \item not gonna go into details here since this is more suitable for longitudinal data, which is not the focus of our presentation, but the idea is similar to dependent subsamples
            \item similarly, $\boldsymbol{\psi} = \{\boldsymbol{\theta}_{1,2}, \boldsymbol{\theta}_{1,3}, \ldots, \boldsymbol{\theta}_{Q-1,Q} \} = \{\boldsymbol{\theta}_{p,q} : p < q\}$
            \item what if both $n$ and $N$ are large? there is no mention of this in the literature, should we mention this?
        \end{itemize}
    }
    \section{R Packages}

    \subsection{\texttt{lme4} and \texttt{mgcv}}
    \begin{frame}
        \frametitle{\insertsection\ $-$ \insertsubsection}
        \texttt{lme4}
        \begin{itemize}
            \item an R package for fitting linear and generalize linear mixed-effects (multilevel) models \footcite{lme4}
            \item efficient, able to handle large sample sizes for simple model, and process hundreds of thousands observations on a typical laptop
            \item Modeling functions: \texttt{lmer()} and \texttt{glmer()}
        \end{itemize}

        \texttt{mgcv}
        \begin{itemize}
            \item an R package for fitting generalized additive model and generalized additive mixed models \footcite{mgcv}
            \item Modeling functions: \texttt{gam()} and \texttt{bam()}
        \end{itemize}
    \end{frame}

    \note{
        \tiny
        can start by saying we have two packages that are used for large datasets, that is lme4 and mgcv

        lme4:
        \begin{itemize}
            \item is an R package for fitting linear and generalized linear mixed-effects (multilevel) models using `Eigen' C\texttt{++} library and S4 classes. (or just say using C++ library). Eigen is a high-level C++ template library for linear algebra that provides efficient, header-only classes for managing matrices, vectors, and numerical solvers. S4 is a formal system in R for object-oriented programming that uses strictly defined classes and methods to ensure data integrity and facilitate complex statistical modeling.
            \item It's computationally efficient, enabling it to handle very large sample sizes for simpler mixed models and to process hundreds of thousands of observations with random effects on a typical laptop.
            \item \texttt{lmer()}: fits linear multilevel model using restricted maximum likelihood (REML) or maximum likelihood estimation. \texttt{glmer()}: fits generalized linear multilevel model, accommodating non-normal response distributions. basically, \texttt{lmer()} for linear models and \texttt{glmer()} for GLM
        \end{itemize}

        mgcv:
        \begin{itemize}
            \item is an R package for fitting generalized additive models (GAMs) and generalized additive mixed models (GAMMs) using penalized regression splines.
            \item \texttt{gam()}: fits generalized additive multilevel models using penalized regression splines with smooth terms that can incorporate multilevel structure through random effect splines. \texttt{bam()}: a computationally efficient version of \texttt{gam()} optimized for very large datasets.
        \end{itemize}
    }

    \subsection{Why use \texttt{bam()}?}
    \begin{frame}
        \frametitle{\insertsection\ $-$ \insertsubsection}
        \begin{itemize}
            \item The underlying model between \texttt{gam()} function and \texttt{lme4} is the same, with differences only in the way parameters are estimated.
            \item \texttt{bam()} employs parallelized computation on model matrix subsets and optional data discretization to extract minimal necessary information, enabling efficient estimation of large multilevel models.
            \item With large enough datasets, this discretization has negligible impact on parameter estimates (differing only at high decimal precision), but leads to dramatic speed improvements.
        \end{itemize}
    \end{frame}

    \subsection{When to use \texttt{bam()}?}
    \begin{frame}
        \frametitle{\insertsection\ $-$ \insertsubsection}
        \begin{itemize}
        \item In general, \texttt{lme4} is preferred for most multilevel datasets due to its straightforward syntax and robust estimation methods.
        \item \texttt{bam()} is particularly useful when:
        \begin{itemize}
            \item The model structure is complex and computationally demanding for \texttt{lme4}
            \item Smooth (nonlinear) terms are required
            \item The dataset is very large and memory limitations arise
            \item Parallel computing resources can be effectively utilized
        \end{itemize}
    \end{itemize}
    \end{frame}
    \section{Summary}

    \begin{frame}
        \frametitle{\insertsection}
        \begin{itemize}
            \item Large multilevel datasets pose significant computational challenges
            \item The split-sample approach offers a practical solution
            \item R packages like \texttt{lme4} and \texttt{mgcv} provide robust tools for fitting multilevel models
            \item Approach and tools selection depends on dataset and research questions
        \end{itemize}
    \end{frame}

    \note{
        \begin{itemize}
            \item However, large multilevel datasets pose significant computational challenges, including memory constraints and slow estimation times.
            \item The split-sample approach offers a practical solution by partitioning data into manageable subsamples, enabling efficient model fitting while retaining key statistical properties.
            \item R packages like \texttt{lme4} and \texttt{mgcv} provide robust tools for fitting multilevel models, with \texttt{bam()} in \texttt{mgcv} being particularly suited for very large datasets due to its computational efficiency.
            \item Choosing the right approach and tools depends on the specific characteristics of the dataset and the research questions at hand.
        \end{itemize}
    }

    % \section{GPT OUTLINE SUGGESTION}

    %     \begin{frame} \scriptsize
    %         \frametitle{OUTLINE FROM GPT - ADJUST ACCORDINGLY}

    %         \begin{columns}
    %             \column{0.5\textwidth}
    %                 2. Introduction (2 min)
    %                 \begin{itemize}
    %                     \item What multilevel (hierarchical) models are
    %                     \item Why they matter in modern statistical modeling
    %                     \item The challenge: traditional multilevel modeling vs massive datasets
    %                     \item Motivating examples (e.g., education, healthcare, online platforms)
    %                 \end{itemize}

    %                 3. Challenges of Multilevel Modeling with Big Data (3 min)
    %                 \begin{itemize}
    %                     \item Memory and computational constraints
    %                     \item Slow estimation with complex models
    %                     \item High-dimensional random effects
    %                     \item Crossed vs nested structures
    %                     \item Sparsity and unbalanced group sizes
    %                     \item Need for distributed or approximate solutions
    %                 \end{itemize}

    %             \column{0.5\textwidth}
    %                 4. Core Approaches to Scaling Multilevel Models (Main Section) (10 min)

    %                 4.1 Approximate Inference Methods
    %                 \begin{itemize}
    %                     \item Variational inference
    %                     \item Laplace approximation
    %                     \item Integrated Nested Laplace Approximation (INLA)
    %                     \item Strengths and trade-offs
    %                 \end{itemize}

    %                 4.2 Penalized and Regularized Models
    %                 \begin{itemize}
    %                     \item L1/L2 penalties on random effects
    %                     \item Elastic net for hierarchical data
    %                     \item Shrinkage approaches for high-dimensional random effects
    %                 \end{itemize}

    %                 4.3 Subsampling and Data Reduction
    %                 \begin{itemize}
    %                     \item Coresets
    %                     \item Sketching methods
    %                     \item Strategic subsampling (e.g., cluster-aware subsampling)
    %                     \item When they work vs when they fail
    %                 \end{itemize}
    %         \end{columns}
    %     \end{frame}

    %     \begin{frame} \scriptsize
    %         \frametitle{OUTLINE FROM GPT}

    %         \begin{columns}
    %             \column{0.5\textwidth}
    %                 4.4 Distributed and Parallel Computing Approaches
    %                 \begin{itemize}
    %                     \item MapReduce-style estimation
    %                     \item Divide-and-recombine (D\&R)
    %                     \item Distributed MCMC and expectation-maximization (EM)
    %                     \item GPU and cloud-based ML
    %                 \end{itemize}

    %                 4.5 Specialized Algorithms for Large Hierarchical Data
    %                 \begin{itemize}
    %                     \item Stochastic gradient descent for multilevel models
    %                     \item Approximate MCMC (e.g., stochastic gradient Langevin dynamics)
    %                     \item Cluster-based aggregations
    %                     \item Sparse matrix optimizations
    %                 \end{itemize}

    %             \column{0.5\textwidth}
    %                 5. Case Studies (2:30 min)

    %                 TBA we can use synthetic data for this i think, check this link \url{https://m-clark.github.io/posts/2019-10-20-big-mixed-models/\#r-packages-for-mixed-models-with-large-data}
    %                 \begin{itemize}
    %                     \item Example 1: Education dataset with millions of students/classrooms
    %                     \item Example 2: Healthcare records (patients nested in hospitals)
    %                     \item Example 3: Online platform data (users nested in geographic regions)
    %                     \item Show results of using an approximate or distributed method
    %                 \end{itemize}

    %         \end{columns}
    %     \end{frame}

    %     \begin{frame} \scriptsize
    %         \frametitle{OUTLINE FROM GPT}

    %         \begin{columns}
    %             \column{0.5\textwidth}
    %                 6. Practical Recommendations (1 min)
    %                 \begin{itemize}
    %                     \item When to choose approximate inference
    %                     \item When distributed computation is necessary
    %                     \item Modeling trade-offs: speed vs accuracy vs complexity
    %                     \item Tips for practitioners
    %                 \end{itemize}

    %             \column{0.5\textwidth}
    %                 7. Conclusion (1 min)
    %                 \begin{itemize}
    %                     \item Multilevel models are powerful but computationally demanding
    %                     \item Scalable solutions make them feasible for modern big data
    %                     \item Importance of balancing interpretability and computational efficiency
    %                 \end{itemize}

    %         \end{columns}
    %     \end{frame}

        % \begin{frame}
        %     \frametitle{Section 1}
        %     This is a text in second frame. For the sake of showing an example.

        %     \begin{itemize}
        %         \item<1-> Text visible on slide 1
        %         \item<2-> Text visible on slide 2
        %         \item<3> Text visible on slides 3
        %         \item<4-> Text visible on slide 4
        %     \end{itemize}
        % \end{frame}

        % \begin{frame}
        %     \frametitle{Basic Introduction}  % Title changes too!
        %     % \frametitle<2-3>{Detailed Analysis}
        %     % \frametitle<4>{Conclusions}

        %     % Text with overlays
        %     This text is always visible.
        %     \only<2->{This text appears from slide 2 onwards.}

        %     % Blocks with overlays
        %     \begin{block}<1>{First Concept}
        %     This block appears only on slide 1.
        %     \end{block}

        %     \begin{block}<2-3>{Analysis Block}
        %     This block appears on slides 2 and 3.
        %     \end{block}

        %     \begin{block}<4>{Final Thoughts}
        %     This block appears only on slide 4.
        %     \end{block}

        %     % Itemize with overlays
        %     \begin{itemize}
        %         \item<1-> Always visible item
        %         \item<2-> Appears on slide 2
        %         \item<3-> Appears on slide 3
        %     \end{itemize}
        % \end{frame}

        %---------------------------------------------------------


        %---------------------------------------------------------
        % \begin{frame}
        % In this slide \pause[]
        % the text will be partially visible \pause[]

        % And finally everything will be there
        % \end{frame}
        %---------------------------------------------------------

    % \section{Second section}

%---------------------------------------------------------
% \begin{frame}
% \frametitle{Sample frame title}

% In this slide, some important text will be
% \alert{highlighted} because it's important.
% Please, don't abuse it.

% \begin{block}{Remark}
% Sample text
% \end{block}

% \begin{alertblock}{Important theorem}
% Sample text in red box
% \end{alertblock}

% \begin{examples}
% Sample text in green box. The title of the block is ``Examples``.
% \end{examples}
% \end{frame}
%---------------------------------------------------------


%---------------------------------------------------------
%Two columns
% \begin{frame}
% \frametitle{Two-column slide}

% \begin{columns}

% \column{0.5\textwidth}
% This is a text in first column.
% \[E=mc^2\]

% \begin{itemize}
% \item First item
% \item Second item
% \end{itemize}

% \column{0.5\textwidth}
% This text will be in the second column
% and on a second tought this is a nice looking
% layout in some cases.
% \end{columns}
% \end{frame}
%---------------------------------------------------------


\end{document}