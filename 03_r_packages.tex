\section{R Packages}

    \subsection{\texttt{lme4}}
    \begin{frame}
        \frametitle{\insertsection\ $-$ \insertsubsection}
        \begin{itemize}
            \item \texttt{lme4} is an R package for fitting linear and generalized linear mixed-effects (multilevel) models using `Eigen' C\texttt{++} library and S4 classes.
            \item It's computationally efficient, enabling it to handle very large sample sizes for simpler mixed models and to process hundreds of thousands of observations with random effects on a typical laptop.
            \item Modeling functions:
                \begin{itemize}
                    \item \texttt{lmer()}: fits linear multilevel model using restricted maximum likelihood (REML) or maximum likelihood estimation.
                    \item \texttt{glmer()}: fits generalized linear multilevel model, accommodating non-normal response distributions.
                \end{itemize}
            \item Reference: \url{https://cran.r-project.org/web/packages/lme4/lme4.pdf}
        \end{itemize}
    \end{frame}

    \subsection{\texttt{mgcv}}
    \begin{frame}
        \frametitle{\insertsection\ $-$ \insertsubsection}
        \begin{itemize}
            \item \texttt{mgcv} is an R package for fitting generalized additive models (GAMs) and generalized additive mixed models (GAMMs) using penalized regression splines.
            \item Modeling functions:
                \begin{itemize}
                    \item \texttt{gam()}: fits generalized additive multilevel models using penalized regression splines with smooth terms that can incorporate multilevel structure through random effect splines.
                    \item \texttt{bam()}: a computationally efficient version of \texttt{gam()} optimized for very large datasets.
                    % fits generalized additive multilevel models to very large datasets, optimized through parallelization and discretization methods.
                \end{itemize}
            \item Reference: \url{https://cran.r-project.org/web/packages/mgcv/mgcv.pdf}
        \end{itemize}
    \end{frame}

    \subsection{Why use \texttt{bam()}?}
    \begin{frame}
        \frametitle{\insertsection\ $-$ \insertsubsection}
        \begin{itemize}
            \item The underlying model between \texttt{gam()} function and \texttt{lme4} is the same, with differences only in the way parameters are estimated.
            \item \texttt{bam()} employs parallelized computation on model matrix subsets and optional data discretization to extract minimal necessary information, enabling efficient estimation of large multilevel models.
            \item With large enough datasets, this discretization has negligible impact on parameter estimates (differing only at high decimal precision), but leads to dramatic speed improvements.
        \end{itemize}
    \end{frame}

    \subsection{When to use \texttt{bam()}?}
    \begin{frame}
        \frametitle{\insertsection\ $-$ \insertsubsection}
        \begin{itemize}
        \item In general, \texttt{lme4} is preferred for most multilevel datasets due to its straightforward syntax and robust estimation methods.
        \item \texttt{bam()} is particularly useful when:
        \begin{itemize}
            \item The model structure is complex and computationally demanding for \texttt{lme4}
            \item Smooth (nonlinear) terms are required
            \item The dataset is very large and memory limitations arise
            \item Parallel computing resources can be effectively utilized
        \end{itemize}
    \end{itemize}
    \end{frame}