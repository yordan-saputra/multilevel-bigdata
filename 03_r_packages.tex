\section{R Packages}

    \subsection{\texttt{lme4} and \texttt{mgcv}}
    \begin{frame}
        \frametitle{\insertsection\ $-$ \insertsubsection}
        \texttt{lme4}
        \begin{itemize}
            \item an R package for fitting linear and generalize linear mixed-effects (multilevel) models \footcite{lme4}
            \item efficient, able to handle large sample sizes for simple model, and process hundreds of thousands observations on a typical laptop
            \item Modeling functions: \texttt{lmer()} and \texttt{glmer()}
        \end{itemize}

        \texttt{mgcv}
        \begin{itemize}
            \item an R package for fitting generalized additive model and generalized additive mixed models \footcite{mgcv}
            \item Modeling functions: \texttt{gam()} and \texttt{bam()}
        \end{itemize}
    \end{frame}

    \note{
        \tiny
        can start by saying we have two packages that are used for large datasets, that is lme4 and mgcv

        lme4:
        \begin{itemize}
            \item is an R package for fitting linear and generalized linear mixed-effects (multilevel) models using `Eigen' C\texttt{++} library and S4 classes. (or just say using C++ library). Eigen is a high-level C++ template library for linear algebra that provides efficient, header-only classes for managing matrices, vectors, and numerical solvers. S4 is a formal system in R for object-oriented programming that uses strictly defined classes and methods to ensure data integrity and facilitate complex statistical modeling.
            \item It's computationally efficient, enabling it to handle very large sample sizes for simpler mixed models and to process hundreds of thousands of observations with random effects on a typical laptop.
            \item \texttt{lmer()}: fits linear multilevel model using restricted maximum likelihood (REML) or maximum likelihood estimation. \texttt{glmer()}: fits generalized linear multilevel model, accommodating non-normal response distributions. basically, \texttt{lmer()} for linear models and \texttt{glmer()} for GLM
        \end{itemize}

        mgcv:
        \begin{itemize}
            \item is an R package for fitting generalized additive models (GAMs) and generalized additive mixed models (GAMMs) using penalized regression splines.
            \item \texttt{gam()}: fits generalized additive multilevel models using penalized regression splines with smooth terms that can incorporate multilevel structure through random effect splines. \texttt{bam()}: a computationally efficient version of \texttt{gam()} optimized for very large datasets.
        \end{itemize}
    }

    \subsection{Why use \texttt{bam()}?}
    \begin{frame}
        \frametitle{\insertsection\ $-$ \insertsubsection}
        \begin{itemize}
            \item The underlying model between \texttt{gam()} function and \texttt{lme4} is the same, with differences only in the way parameters are estimated.
            \item \texttt{bam()} employs parallelized computation on model matrix subsets and optional data discretization to extract minimal necessary information, enabling efficient estimation of large multilevel models.
            \item With large enough datasets, this discretization has negligible impact on parameter estimates (differing only at high decimal precision), but leads to dramatic speed improvements.
        \end{itemize}
    \end{frame}

    \subsection{When to use \texttt{bam()}?}
    \begin{frame}
        \frametitle{\insertsection\ $-$ \insertsubsection}
        \begin{itemize}
        \item In general, \texttt{lme4} is preferred for most multilevel datasets due to its straightforward syntax and robust estimation methods.
        \item \texttt{bam()} is particularly useful when:
        \begin{itemize}
            \item The model structure is complex and computationally demanding for \texttt{lme4}
            \item Smooth (nonlinear) terms are required
            \item The dataset is very large and memory limitations arise
            \item Parallel computing resources can be effectively utilized
        \end{itemize}
    \end{itemize}
    \end{frame}